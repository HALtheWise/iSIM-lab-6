% This is a simple template for an iSIM LaTeX lab using the "article" class. Read both the text and the comments--everything is useful!

\documentclass[11pt]{article} % use larger type; default would be 10pt

% There's a lot of boilerplate crap that you don't need to understand
\usepackage[utf8]{inputenc}

% These are packages you may need... It doesn't cost anything to include them all, so you might as well.
\usepackage{amstext} %allows you to put text in math mode
\usepackage{amsmath} %includes lots of math-related capabilities
\usepackage{graphicx} %allows you to include pictures
\usepackage{float} %improves the use of floating objects (like picutes)
\usepackage{caption} %allows you to change caption styles on figures
\usepackage{epstopdf} %automatically converts EPS files (like from matlab)
\usepackage{hyperref} %allows you to include links
\usepackage{varioref} %requirement of fancyref
\usepackage{fancyref} %allows really nice looking and convienient references
\usepackage[section]{placeins} %makes your figures not float past section barriers
\usepackage{perpage} %restarts footnote numbering by page
\usepackage[margin=1in, paperwidth=8.5in, paperheight=11in]{geometry} %I'll bet you can figure this one out
\MakeSorted{figure} %deals with figures using both [h] and [H]

% Makes sure your document compiles when you screw up the references
\vrefwarning

% Uncomment the following line to default all of your figures to [H] (explained later)
%\float­place­ment{fig­ure}{H}

%add some metadata to the PDF produced
 % \pdfinfo{/Author (Eric Miller)

%%% The "real" document content comes below...

\title{Lab Report: Glucose monitor}
\author{Eric Miller}
%\date{Now} % Leave this commented to automatically display the current date. Otherwise, you can redefine it here.

\begin{document}
\maketitle % Make sure to include this (after begin{document}, or you'll have no title!

%\begin{abstract}
%\LaTeX~is a markup language (like HTML) that allows you to create beautiful lab reports (among other things) with a wide variety of STEM-oriented features. Like almost everything else, you can find out a lot about using \LaTeX~by using Google. You can also have a look at the wikibook\footnote{You can add links like so: \href{http://en.wikibooks.org/wiki/LaTeX}{\LaTeX~Wikibook} Have a look at that wikibook--it contains almost everything you'll ever need to know about \LaTeX.} which is usually among the first few Google results for most \LaTeX-related queries.
%\end{abstract}

\section{Voltage Source}

Having my younger siblings visiting for Family Weekend, I asked them to find a pair of resistors ending in "k" with a ratio of approximately $2:3$. The resulting resistor values were nominally $4 k\Omega$ and $6.04 k\Omega$, for a theoretical voltage divider output of 
$$5V*\frac {4 k\Omega} {4 k\Omega + 6.04 k\Omega} = 1.992 V$$

The actual measured value was slightly lower, at $1.97V$, but the difference wasn't significant enough to be a problem.

%Note the use of \Fref here... This automatically creates a reference to your figure, no matter where it is in the document. If it is on a different page, \Fref will include the page of the figure. For this to work, you have to provide intelligent labels, i.e. fig:whatever for figures. Labels should be lowercase and have no spaces.	

%the [!ht] here means "LaTeX, I would very much like you to put this figure here, but if you can't that's ok, just make sure it's at the top of some other page"
% However, because you (what a smart cookie you are) decided to include the float package, you can use the option [H], which tells LaTeX "PUT THE ****ing FIGURE HERE OR I WILL GO BACK TO USING WORD YOU GODDAMN STUPID PIECE OF **** TYPESETTING PROGRAM" 
\begin{figure}[H]
	\centering
	\includegraphics[width=.7\textwidth]{CD1.PNG}
	\caption{Voltage Divider circuit}
\end{figure}

\begin{figure}[H]
	\centering
	%instead of entering the width in terms of textwidth, you can use a set number of inches, or pt, or mm, or whatever
	\includegraphics[width=.7\textwidth]{VoltageDivider.png}
	%Make sure to caption your figures! LaTeX will automatically number them for you.
	\caption{Voltage Divider results}
 	\label{fig:awesome}
\end{figure}

%You can make subsections (and subsubsections) if you need to go into detail
\section{Measure Resistance}
 !!! NEED THEORY !!!
Five resistor values were tested, as shown below. For each resistor, the current it theoretically allowed was calculated as $\frac {1.976V} {[Resistance]}$, also in the table.
\begin{table}[H]
\centering
\caption{Resistor testing data}
\begin{tabular}{lll}
Resistance & Calculated Current (mA) & Measured voltage \\
604        & 3.272                   & 4.96             \\
4000       & 0.494                   & 3.75             \\
20000      & 0.099                   & 2.74             \\
60000      & 0.033                   & 2.54             \\
249000     & 0.008                   & 2.50            
\end{tabular}
\end{table}

Shown here is a plot of the Currents and measurements for the lower four rows in the table, those with less than $1mA$ theoretical current flow. As expected, the data points fall nicely along a line, giving an accurate way to measure small currents. (\Fref{fig:some})

Unfortunately, the $604\Omega$ resistor proved to allow more current to pass than the system could reliably measure, as seen by the fact that its data point falls well off the line. (\Fref{fig:current})



\begin{figure}[]
	\centering
	\includegraphics[width=\textwidth]{4points-current.png}
	\caption{Partial current measurement results}
 	\label{fig:some}
\end{figure}

\begin{figure}[]
	\centering
	%instead of entering the width in terms of textwidth, you can use a set number of inches, or pt, or mm, or whatever
	\includegraphics[width=\textwidth]{5points-current.png}
	%Make sure to caption your figures! LaTeX will automatically number them for you.
	\caption{Full current measurement results}
 	\label{fig:current}
\end{figure}


\section{Integrator }

In \fref{fig:integrate}, the orange line charts the input to the circuit, and the blue line charts the output. Clearly, when the orange line is HIGH, the blue line rises and inversely. This is the expected behavior of an integrator.

\begin{figure}[H]
	\centering
	%instead of entering the width in terms of textwidth, you can use a set number of inches, or pt, or mm, or whatever
	\includegraphics[width=\textwidth]{integrator.png}
	%Make sure to caption your figures! LaTeX will automatically number them for you.
	\caption{Integrator testing results}
 	\label{fig:integrate}
\end{figure}



\section{Glucose sensor}


 The glucose monitor, as expected, produced substantially higher currents when the concentration of the glucose in the solution was higher. The raw data collected can be seen in
 \fref{fig:gcurrent}, where concentrations are measured in mg/dL. \Fref{fig:gcal} shows some values extracted from these curves, including the maximum amount of current measured, the current flow measured after 10 seconds, and the value of the integrator after 10 seconds (adjusted for the starting value).
 
 These results indicate that the maximum amount of current is a poor indicator of the concentration, with the points substantially deviating from a line. The integral of current is better, but still does not display quite as much regularity as the absolute current measurement after 10 seconds. This could be a measurement peculiarity from this specific trial. Further investigation is needed to determine whether some combination of these signals could be used to create an even more consistent measurement.

\begin{figure}[]
	\centering
	%instead of entering the width in terms of textwidth, you can use a set number of inches, or pt, or mm, or whatever
	\includegraphics[width=\textwidth]{glucose_currents.png}
	%Make sure to caption your figures! LaTeX will automatically number them for you.
	\caption{Time series data for various glucose concentrations}
 	\label{fig:gcurrent}
\end{figure}

\begin{figure}[]
	\centering
	%instead of entering the width in terms of textwidth, you can use a set number of inches, or pt, or mm, or whatever
	\includegraphics[width=\textwidth]{glucose_calibration.png}
	%Make sure to caption your figures! LaTeX will automatically number them for you.
	\caption{}
 	\label{fig:gcal}
\end{figure}





\end{document}



